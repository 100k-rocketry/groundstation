\documentclass[10pt,draftclsnofoot,onecolumn]{IEEEtran}
\hyphenation{op-tical net-works semi-conduc-tor}

\usepackage[margin=.75in]{geometry}
\usepackage{courier}
\usepackage{ifthen}
\usepackage{setspace}
\usepackage{listings}
\usepackage[usenames, dvipsnames]{color}
\usepackage{tabularx}
\usepackage[strict]{chngpage}
\usepackage{cite}
\usepackage{graphicx}
\usepackage{acronym}
\usepackage{color}
\usepackage{makeidx}
\usepackage{url}

\makeindex

\acrodef{OSU}[OSU]{Oregon State University}
\acrodef{AIAA}[AIAA]{American Institute of Aeronautics and Astronautics}
\acrodef{AGL}[AGL]{Above Ground Level}
\acrodef{JSON}[JSON]{JavaScript Object Notation}

\lstset {
	language=C,
	basicstyle=\ttfamily,
	keywordstyle=\color{blue}\ttfamily,
	stringstyle=\color{red}\ttfamily,
	commentstyle=\color{OliveGreen}\ttfamily,
	morecomment=[l][\color{magenta}]{\#}
	showstringspaces=false,
	showspaces=false,
	frame=single,
	captionpos=b
}

\newcommand{\commandline}[2][\empty] {
	\begin{quote}
		\texttt{#2}
		\ifthenelse{\equal{#1}{\empty}}{}{\begin{quote}#1\end{quote}}
	\end{quote}
}

\newcommand{\sigline}[1][\empty] {
	\vspace{1in}
	\hrule width0.5\textwidth
	\vspace{1mm}
	\noindent #1	
}

\newcommand*{\SignatureAndDate}[1]{
	\vspace{1in}
	\par\noindent\makebox[2.5in]{\hrulefill} \hspace{.5in} \makebox[2.0in]{\hrulefill}
	\par\noindent\makebox[2.5in][l]{#1}      \hspace{.5in} \makebox[2.0in][l]{Date}
}



\begin{document}
	\singlespace
	
	\title{\vspace{2in}Design}
	
	\author {
		Anisimova, Natasha
		\and
		Lee, Terrance
		\and
		Morgan, Albert
	}
	
	\markboth{CS Capstone 2016-2017}{Groundstation}
	
	\pagestyle{empty}
	\vspace*{2in}
	\begin{center}
		\huge
		Groundstation: Design\\
		\normalsize
		\vspace{5mm}
		\textbf{
			Team \#25\\
			High-Altitude Rocketry Challenge\\
		}
		\vspace{1mm}
		Natasha Anisimova\\
		Terrance Lee\\
		Albert Morgan
	\end{center}
	
	\vspace{5mm}
	
	\begin{center}
		\textbf{Abstract}
	\end{center}
	
	%\begin{adjustwidth}{0.75in}{0.75in}
	
	%Abstract goes here
	The \textit{Groundstation} software will collect telemetry from a rocket while is in flight and graphically display the telemetry in real-time. Groundstation is made up several different components: collection of data, storage of data, interpolation of data, and 
	display of data.	
	This document will examine nine different components of the system.
	For each of these components, three different technologies will be described and evaluated for use in this component.
	Finally, a recommendation will be made about which technology should be used.	
	%\end{adjustwidth}
	
	\newpage

	% Uncomment this to make the table of contents	
	%\tableofcontents
	%\newpage

	\section{Introduction}

	\begin{enumerate}
	\item Identitication of the SSD
	\item stakeholders
	\item concerns
	\item selected viewpoints
	\item design views
	\item design overlays
	\item rationale
	\end{enumerate}

	Maybe we don't need all of this stuff in the intro? I think some of it is covered below.

	
	
	\section{Structure Viewpoint}
	
	\subsection{Entity}
	\subsubsection{Name}
	Web server.
	\subsubsection{Type}
	Program
	\subsubsection{Contents}
	The web server will run on the Raspberry Pi. The web server has three primary functions:
	\begin{itemize}
	\item Server web pages to the clients.
	\item Receive telemetry from the serial port and convert it into json.
	\item Make the json data available to the clients.
	\end{itemize}
	Groundstation will use the Apache~\cite{Apache} web server.	
	
	
	
	
	
	\vspace{1in}
	{\Huge Everything after this is just brainstorming}
	
	
	
	\section{Context}

	Here's a section for the context design viewpoint.
	Context covers systems services and users.
	It has the following subsections:

	\subsection{Concerns}
	Design concerns come from the requirements. So, things like "The users need to visualize the data in real-time"
	and "We need to get telemetry from the rocket in real-time".

	There are four types of elements: entity, relationship, attribute, constraint

	\subsection{Example entity}
	Entities are library, a system, framework, class, etc... It has the following sections
	\begin{itemize}
		\item Type: Framework, library, class, object, etc...
		\item Purpose: Why does this entity exist?
	\end{itemize}



	\subsection{Example relationship}
	We probably want a lot of these.
	They describe how entities work together.
	One might be "We get data from the rocket, and it goes to the processing entity".
	Another might be "We sent th eprocessed data to the web browser".


	\subsection{Attribute is a fact}
	They usually answer questions, and are sort-of in Q\&A format.
	What language are we using? Answer: we're going to use Node. Like that.

	\subsection {Constraint}
	``There ain't no Internet in the desert'' and stuff like that.



	\subsection{Example name}


	\subsection{Elements}
	This is a list of entities, relationships, attibutes, or constraints


	\subsection{Methods}

	\subsection{Source}
	
	If we need to cite something, it goes in here.



	\section{Interaction}
	Talk about how the system will get data from the serial port and how it will get sent to the web browser.

	\section{Algorithm}
	Stuff about the event-driven architecture maybe.


	
	\pagestyle{headings}

	% CONTENT GOES HERE
	
%	\printindex

\bibliography{groundstation}
\bibliographystyle{IEEEtran}

\newpage
\begin{minipage}{\textwidth}	
	\SignatureAndDate{Nancy Squires}
	\SignatureAndDate{Natasha Anisimova}
	\SignatureAndDate{Terrance Lee}
	\SignatureAndDate{Albert Morgan}
\end{minipage}

\end{document}
