\documentclass[10pt,draftclsnofoot,onecolumn]{IEEEtran}
\hyphenation{op-tical net-works semi-conduc-tor}

\usepackage[margin=.75in]{geometry}
\usepackage{courier}
\usepackage{ifthen}
\usepackage{setspace}
\usepackage{listings}
\usepackage[usenames, dvipsnames]{color}
\usepackage{tabularx}
\usepackage[strict]{chngpage}
\usepackage{cite}
\usepackage{graphicx}
\usepackage{acronym}
\usepackage{color}
\usepackage{makeidx}
\usepackage{url}

\makeindex

\acrodef{NPM}[NPM]{Node Package Manager}

\lstset {
	language=C,
	basicstyle=\ttfamily,
	keywordstyle=\color{blue}\ttfamily,
	stringstyle=\color{red}\ttfamily,
	commentstyle=\color{OliveGreen}\ttfamily,
	morecomment=[l][\color{magenta}]{\#}
	showstringspaces=false,
	showspaces=false,
	frame=single,
	captionpos=b
}

\newcommand{\commandline}[2][\empty] {
	\begin{quote}
		\texttt{#2}
		\ifthenelse{\equal{#1}{\empty}}{}{\begin{quote}#1\end{quote}}
	\end{quote}
}

\newcommand{\sigline}[1][\empty] {
	\vspace{1in}
	\hrule width0.5\textwidth
	\vspace{1mm}
	\noindent #1	
}

\newcommand*{\SignatureAndDate}[1]{
	\vspace{1in}
	\par\noindent\makebox[2.5in]{\hrulefill} \hspace{.5in} \makebox[2.0in]{\hrulefill}
	\par\noindent\makebox[2.5in][l]{#1}      \hspace{.5in} \makebox[2.0in][l]{Date}
}

\newcommand{\newentity}[5]{
	\noindent\textbf{Entity} (Author: {#1})
	
	\noindent\textit{Name:} {#2}
	
	\noindent\textit{Type:} {#3}
	
	\noindent\textit{Purpose:} {#4}
	
	\noindent\textit{Contents:} {#5}
	\vspace{.5cm}
}

\newcommand{\newrelationship}[4]{
	\noindent\textbf{Relationship} (Author: #1)

	\noindent\textit{Name:} #2

	\noindent\textit{Type:} #3

	\noindent\textit{Contents:} #4
	\vspace{.5cm}
}

\newcommand{\newconstraint}[6]{
	\subsection{Constraint} (Author: #1)
	\subsubsection{Name}
	#2
	\subsubsection{Type}
	#3
	\subsubsection{Source}
	#4
	\subsubsection{Target}
	#5
	\subsubsection{Contents}
	#6
}

\newcommand{\newattribute}[5]{
	\subsection{Attribute} (Author: #1)
	\subsubsection{Name}
	#2
	\subsubsection{Type}
	#3
	\subsubsection{Purpose}
	#4
	\subsubsection{Contents}
	#5	
}




\begin{document}
	\singlespace
	
	\title{\vspace{2in}Design}
	
	\author {
		Anisimova, Natasha
		\and
		Lee, Terrance
		\and
		Morgan, Albert
	}
	
	\markboth{CS Capstone 2016-2017}{Groundstation}
	
	\pagestyle{empty}
	\vspace*{2in}
	\begin{center}
		\huge
		Groundstation: Design\\
		\normalsize
		\vspace{5mm}
		\textbf{
			Team \#25\\
			High-Altitude Rocketry Challenge\\
		}
		\vspace{1mm}
		Natasha Anisimova\\
		Terrance Lee\\
		Albert Morgan
	\end{center}
	
	\vspace{5mm}
	
	\begin{center}
		\textbf{Abstract}
	\end{center}
	
	%\begin{adjustwidth}{0.75in}{0.75in}
	
	%Abstract goes here
	The \textit{Groundstation} software will collect telemetry from a rocket while is in flight and graphically display the telemetry in real-time. Groundstation is made up several different components: collection of data, storage of data, interpolation of data, and 
	display of data.	
	This document will examine nine different components of the system.
	For each of these components, three different technologies will be described and evaluated for use in this component.
	Finally, a recommendation will be made about which technology should be used.	
	%\end{adjustwidth}
	
	\newpage

	% Uncomment this to make the table of contents	
	%\tableofcontents
	%\newpage

	\section{Introduction}

	\begin{enumerate}
	\item Identitication of the SSD
	\item stakeholders
	\item concerns
	\item selected viewpoints
	\item design views
	\item design overlays
	\item rationale
	\end{enumerate}

	Maybe we don't need all of this stuff in the intro? I think some of it is covered below.

	
	
	\section{Structure Viewpoint}
		
	
	
	% LOOK AT MY NEWELEMENT COMMAND EXAMPLEHERE
	

	\newentity
	{Your name here}
	{PC}
	{Component}
	{Why does it exists?}
	{Stuff here.}

	\newentity
	{Albert Morgan}
	{Package Manager}
	{Subprogram}
	{
		The package manager will install, track, and update software dependencies on the server.
	}
	{
		Because Groundstation will be using Node and JavaScript for both the frontend and backend, \ac{NPM} will be used~\cite{npm}. \ac{NPM} has a large repository of both server-side and client-side JavaScript packages.
	}

	\newentity
	{Your name here}
	{Frontend}
	{Component}
	{User interface}
	{Stuff here.}

	\newentity
	{Your name here}
	{Backend}
	{Component}
	{Web server software stuff}
	{Stuff here.}

	\newentity
	{Your name here}
	{Node}
	{Subprogram}
	{Runs the backend}
	{Stuff here.}

	\newentity
	{Your name here}
	{Serialport}
	{Library}
	{Node serialport library}
	{Stuff here.}

	\newentity
	{Your name here}
	{Log}
	{Data store}
	{This is where data gets logged}
	{Stuff here.}
	
	
	\newentity
	{Your name here}
	{jQuery}
	{Libary}
	{UI stuff}
	{Queries the J}
	
	\newentity
	{Your name here}
	{3.js}
	{Libary}
	{UI stuff}
	{All of your 3 needs}


	\newentity
	{Your name here}
	{Rocket}
	{Component}
	{Gets high}
	{ZOOM}
	
	\newrelationship
	{Your name here}
	{jQuery}
	{Libary}
	{UI stuff}
	{Queries the J}

	\newentity
	{Albert Morgan}
	{Web server}
	{Process}
	{	The web server will serve three primary functions:
		\begin{itemize}
		\item Server web pages to the clients.
		\item Receive telemetry from the serial port and convert it into json.
		\item Make the json data available to the clients.
		\end{itemize}
	}
	{	The web server will run on the Raspberry Pi. The web server has three primary functions:
		Groundstation will use the Apache~\cite{apache} web server.
	}


	\newentity
	{Albert Morgan}
	{Web browser}
	{Process}
	{The web server}
	{	The client will use a web browser to connect to the Groundstation web server and access the content.
		The web browser may be any of:
		\begin{itemize}
			\item Chrome version 54 or higher
			\item Edge version 14 or higher
			\item Firefox version 49 or higher
			\item Safari version 10 or higher
		\end{itemize}
	}


	\newrelationship
	{Your name here}
	{Web browser composition}
	{Composition}
	{The web browser runs on the PC}
	
	\newrelationship
	{Your name here}
	{Frontend composition}
	{Composition}
	{Stuff}
	
	\newrelationship
	{Your name here}
	{Backend composition}
	{Composition}
	{Stuff}
	
	\newrelationship
	{Your name here}
	{jQuery composition}
	{Composition}
	{Stuff}
	
	\newrelationship
	{Your name here}
	{3.js composition}
	{Composition}
	{Stuff}
	
	\newrelationship
	{Your name here}
	{Node composition}
	{Composition}
	{Stuff}
	
	\newrelationship
	{Your name here}
	{Serialport use}
	{Use}
	{Stuff}
	
	\newrelationship
	{Your name here}
	{Log composition}
	{Composition}
	{Stuff}
	
	\newrelationship
	{Your name here}
	{Web browser use}
	{Use}
	{Uses the web server}
	
	\newrelationship
	{Your name here}
	{Frontend / Backend relationship}
	{Composition}
	{Backend servers frontend}
	
	\newrelationship
	{Your name here}
	{Backend / Rocket}
	{Use}
	{Gets data from the rocket}
	
	\newrelationship
	{Your name here}
	{NPM / Frontend}
	{Use}
	{Frontend uses NPM}
	
	\newrelationship
	{Your name here}
	{NPM / Backend}
	{Use}
	{Backend uses NPM}

	\newrelationship
	{Your name here}
	{NPM / Backend}
	{Use}
	{Backend uses NPM}

	\section{Interaction}
	Talk about how the system will get data from the serial port and how it will get sent to the web browser.

	\section{Algorithm}
	Stuff about the event-driven architecture maybe.


	
	\pagestyle{headings}

	% CONTENT GOES HERE
	
%	\printindex

\bibliography{groundstation}
\bibliographystyle{IEEEtran}

\newpage
\begin{minipage}{\textwidth}	
	\SignatureAndDate{Nancy Squires}
	\SignatureAndDate{Natasha Anisimova}
	\SignatureAndDate{Terrance Lee}
	\SignatureAndDate{Albert Morgan}
\end{minipage}

\end{document}
