\documentclass[10pt,draftclsnofoot,onecolumn]{IEEEtran}
\hyphenation{op-tical net-works semi-conduc-tor}

\usepackage[margin=.75in]{geometry}
\usepackage{courier}
\usepackage{ifthen}
\usepackage{setspace}
\usepackage{listings}
\usepackage[usenames, dvipsnames]{color}
\usepackage{tabularx}
\usepackage[strict]{chngpage}
\usepackage{cite}
\usepackage{graphicx}
\usepackage{acronym}
\usepackage{color}
\usepackage{makeidx}
\usepackage{url}

\makeindex

\acrodef{NPM}[NPM]{Node Package Manager}
\acrodef{OSU}[OSU]{Oregon State University}
\acrodef{AIAA}[AIAA]{American Institute of Aeronautics and Astronautics}

\lstset {
	language=C,
	basicstyle=\ttfamily,
	keywordstyle=\color{blue}\ttfamily,
	stringstyle=\color{red}\ttfamily,
	commentstyle=\color{OliveGreen}\ttfamily,
	morecomment=[l][\color{magenta}]{\#}
	showstringspaces=false,
	showspaces=false,
	frame=single,
	captionpos=b
}

\newcommand{\commandline}[2][\empty] {
	\begin{quote}
		\texttt{#2}
		\ifthenelse{\equal{#1}{\empty}}{}{\begin{quote}#1\end{quote}}
	\end{quote}
}

\newcommand{\sigline}[1][\empty] {
	\vspace{1in}
	\hrule width0.5\textwidth
	\vspace{1mm}
	\noindent #1	
}

\newcommand*{\SignatureAndDate}[1]{
	\vspace{1in}
	\par\noindent\makebox[2.5in]{\hrulefill} \hspace{.5in} \makebox[2.0in]{\hrulefill}
	\par\noindent\makebox[2.5in][l]{#1}      \hspace{.5in} \makebox[2.0in][l]{Date}
}

\newcommand{\newentity}[5]{
	\begin{minipage}{\linewidth}
	\noindent\textbf{#2}
	
	\noindent Entity
	
	\noindent\textit{Author:} {#1}
		
	\noindent\textit{Type:} {#3}
	
	\noindent\textit{Purpose:} {#4}
	
	\noindent\textit{Contents:} {#5}
	\vspace{.5cm}
	\end{minipage}
}

\newcommand{\newrelationship}[4]{
	\begin{minipage}{\linewidth}
	\noindent\textbf{#2}
	
	\noindent Relationship
	
	\noindent\textit{Author:} #1

	\noindent\textit{Type:} #3

	\noindent\textit{Contents:} #4
	\vspace{.5cm}
	\end{minipage}
}

\newcommand{\newconstraint}[6]{
	\subsection{Constraint} (Author: #1)
	\subsubsection{Name}
	#2
	\subsubsection{Type}
	#3
	\subsubsection{Source}
	#4
	\subsubsection{Target}
	#5
	\subsubsection{Contents}
	#6
}

\newcommand{\newattribute}[5]{
	\subsection{Attribute} (Author: #1)
	\subsubsection{Name}
	#2
	\subsubsection{Type}
	#3
	\subsubsection{Purpose}
	#4
	\subsubsection{Contents}
	#5	
}




\begin{document}
	\singlespace
	
	\title{\vspace{2in}Design}
	
	\author {
		Anisimova, Natasha
		\and
		Lee, Terrance
		\and
		Morgan, Albert
	}
	
	\markboth{CS Capstone 2016-2017}{Groundstation}
	
	\pagestyle{empty}
	\vspace*{2in}
	\begin{center}
		\huge
		Groundstation: Design\\
		\normalsize
		\vspace{5mm}
		\textbf{
			Team \#25\\
			High-Altitude Rocketry Challenge\\
		}
		\vspace{1mm}
		Natasha Anisimova\\
		Terrance Lee\\
		Albert Morgan
	\end{center}
	
	\vspace{5mm}
	
	\begin{center}
		\textbf{Abstract}
	\end{center}
	
	%\begin{adjustwidth}{0.75in}{0.75in}
	
	%Abstract goes here
	The \textit{Groundstation} software will collect telemetry from a rocket while is in flight and graphically display the telemetry in real-time. Groundstation is made up several different components: collection of data, storage of data, interpolation of data, and 
	display of data.	
	This document will describe in detail the design of Groundstation.
	%\end{adjustwidth}
	
	\newpage

	% Uncomment this to make the table of contents	
	%\tableofcontents
	%\newpage

	\section{Introduction}

	\subsection{Identification}
	This document will provide and in-depth software design description of the Groundstation software package.
	
	\subsection{Stakeholders}
	The stakeholder of the Groundstation software package is the \ac{OSU} \ac{AIAA} High-Altitude Rocketry Team.
	This group of multidisciplinary engineering students who are working to together to build the rocket.
	\begin{itemize}
		\item The ability to view the data in real time in order to track the altitude and aid in recovery.
		\item The abilk
	\end{itemize}
	

	\begin{enumerate}
	\item Identitication of the SSD
	\item stakeholders
	\item concerns
	\item selected viewpoints
	\item design views
	\item design overlays
	\item rationale
	\end{enumerate}

	Maybe we don't need all of this stuff in the intro? I think some of it is covered below.

	
	
	\section{Structure View}
		
	
	
	% LOOK AT MY NEWELEMENT COMMAND EXAMPLEHERE
	

	\newentity
	{Your name here}
	{PC}
	{Component}
	{Why does it exists?}
	{Stuff here.}

	\newentity
	{Albert Morgan}
	{Package Manager}
	{Subprogram}
	{
		The package manager will install, track, and update software dependencies on the server.
	}
	{
		Because Groundstation will be using Node and JavaScript for both the frontend and backend, \ac{NPM} will be used~\cite{npm}. \ac{NPM} has a large repository of both server-side and client-side JavaScript packages.
	}

	\newentity
	{Natasha Anisimova}
	{Frontend}
	{Component}
	{User interface}
	{Stuff here.}

	\index{Backend}
	\newentity
	{Albert Morgan}
	{Backend}
	{Component}
	{
		The purpose of the backend is the facilitate communication between the rocket and the user.
	}
	{
		The backend takes care of all data collection, transformation, logging, and serving that data to the user.
		A Node server will handle reading the data from the serial port, transforming the data into JSON,
		storing the data, and making the data available to the user.
	}

	\index{Node}
	\index{Backend}
	\index{PHP}
	\index{Ruby}
	\index{Raspberry Pi}
	\newentity
	{Albert Morgan}
	{Node}
	{Subprogram}
	{Node is the software that runs the backend.}
	{
		The backend will run on Node. Several choices were considered to run the backend, particularly Node, PHP, and Ruby.
		The backend was chosen two criteria:
		\begin{itemize}
			\item Speed. The backend will run on a Raspberry Pi, so speed is important to minimize the system resources used.
			\item Interoperability. The backend needs to work with multiple components, including the logging software, the data coming in from the serial port, serving the frontend to the user.
		\end{itemize}
		Node uses an event driven architecture, which is ideal for reading data from the serial port.
		PHP and Ruby on Rails are both HTML preprocessors, so they don't get activated until the web site is requested.
		Node, on the other hand, runs continuously in the background and uses an event driven architecture.
		The event driven architecture is ideal for reading the data from the serial port.
		Additionally, Node much faster than either ruby or PHP.	
	}

	\newentity
	{Your name here}
	{Serialport}
	{Library}
	{Node serialport library}
	{Stuff here.}

	\index{Log}
	\index{JSON}
	\newentity
	{Albert Morgan}
	{Log}
	{Data store}
	{Store the data in non-volatile storage for later retrieval and update new connecting clients}
	{
		Groundstation will store telemetry in JSON format.
		The log should limit the possibility of the corruption of the data due to a programming error
		and make the data easily accessible for newly connecting clients.
		Several choices were considered for the logging, including relational databases, JSON documents, and logging the telemetry.
		A relational database would be unnecessary because the data does not have any relations that need to be tracked;
		each telemetry packet is an independent piece of information.
		Logging the telemetry as it comes in from the rocket would limit the possibility of corruption.
		A programming error in the JSON parsing routines could cause all of the data to be unusable.
		However, storing the data in JSON format would make it very easy for new clients that connect to the server in the middle of the rocket flight to be updated with all of the past data.
		For this reason, the telemetry will be stored in a JSON format.
	}
	
	
	\newentity
	{Your name here}
	{jQuery}
	{Libary}
	{UI stuff}
	{Queries the J}
	
	\newentity
	{Natasha Anisimova}
	{3.js}
	{Libary}
	{3.js is a JavaScript 3D library used to create and display animated 3D computer graphics using WebGL.}
	{	Since the information about the rocket will be displayed by using a web browser through a local Wi-Fi network, 
		making sure the information is displayed on time and correctly is crucial. 3.js allows for easy and rapid development
		of WebGL applications, which means we can make the most of the specialised graphics hardware on the users'
		PCs that it gives access to.
	}


	\newentity
	{Your name here}
	{Rocket}
	{Component}
	{Gets high}
	{ZOOM}
	
	\index{Web server}
	\index{Raspberry Pi}
	\index{JSON}
	\index{Telemetry}
	\newentity
	{Albert Morgan}
	{Web server}
	{Process}
	{	The web server will serve three primary functions:
		\begin{itemize}
		\item Server web pages to the clients.
		\item Receive telemetry from the serial port and convert it into json.
		\item Make the JSON data available to the clients.
		\end{itemize}
	}
	{	The web server will run on the Raspberry Pi. The web server has three primary functions:
		Groundstation will use the Apache~\cite{apache} web server.
	}


	\index{Web browser}
	\index{Chrome}
	\index{Edge}
	\index{Firefox}
	\index{Safari}
	\newentity
	{Albert Morgan}
	{Web browser}
	{Process}
	{The web server}
	{	The client will use a web browser to connect to the Groundstation web server and access the content.
		The web browser may be any of:
		\begin{itemize}
			\item Chrome version 54 or higher
			\item Edge version 14 or higher
			\item Firefox version 49 or higher
			\item Safari version 10 or higher
		\end{itemize}
	}


	\newrelationship
	{Your name here}
	{Web browser composition}
	{Composition}
	{The web browser runs on the PC}
	
	\newrelationship
	{Natasha Anisimova}
	{Frontend composition}
	{Composition}
	{Stuff}
	
	\newrelationship
	{Your name here}
	{Backend composition}
	{Composition}
	{Stuff}
	
	\newrelationship
	{Your name here}
	{jQuery composition}
	{Composition}
	{Stuff}
	
	\newrelationship
	{Natasha Anisimova}
	{3.js composition}
	{Composition}
	{Stuff}
	
	\newrelationship
	{Your name here}
	{Node composition}
	{Composition}
	{Stuff}
	
	\newrelationship
	{Your name here}
	{Serialport use}
	{Use}
	{Stuff}
	
	\newrelationship
	{Your name here}
	{Log composition}
	{Composition}
	{Stuff}
	
	\newrelationship
	{Your name here}
	{Web browser use}
	{Use}
	{Uses the web server}
	
	\newrelationship
	{Your name here}
	{Frontend / Backend relationship}
	{Composition}
	{Backend servers frontend}
	
	\newrelationship
	{Your name here}
	{Backend / Rocket}
	{Use}
	{Gets data from the rocket}
	
	\newrelationship
	{Your name here}
	{NPM / Frontend}
	{Use}
	{Frontend uses NPM}
	
	\newrelationship
	{Your name here}
	{NPM / Backend}
	{Use}
	{Backend uses NPM}

	\section{Interaction}
	Talk about how the system will get data from the serial port and how it will get sent to the web browser.

	\section{Algorithm}
	Stuff about the event-driven architecture maybe.


	
	\pagestyle{headings}

	% CONTENT GOES HERE
	
%	\printindex

\bibliography{groundstation}
\bibliographystyle{IEEEtran}

\newpage
\begin{minipage}{\textwidth}	
	\SignatureAndDate{Nancy Squires}
	\SignatureAndDate{Natasha Anisimova}
	\SignatureAndDate{Terrance Lee}
	\SignatureAndDate{Albert Morgan}
\end{minipage}

\end{document}
