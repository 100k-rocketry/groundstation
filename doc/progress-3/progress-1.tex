\documentclass[10pt,draftclsnofoot,onecolumn]{IEEEtran}
\hyphenation{op-tical net-works semi-conduc-tor}

\usepackage[margin=.75in]{geometry}
\usepackage{courier}
\usepackage{ifthen}
\usepackage{setspace}
\usepackage{listings}
\usepackage[usenames, dvipsnames]{color}
\usepackage{tabularx}
\usepackage[strict]{chngpage}
\usepackage{cite}
\usepackage{graphicx}
\usepackage{acronym}
\usepackage{color}
\usepackage{makeidx}
\usepackage{url}
\usepackage{listings}
\usepackage{verbatim}

\makeindex

\acrodef{NPM}[NPM]{Node Package Manager}
\acrodef{OSU}[OSU]{Oregon State University}
\acrodef{AIAA}[AIAA]{American Institute of Aeronautics and Astronautics}
\acrodef{COCOM}[COCOM]{Coordinating Committee for Multilateral Export Controls}
\acrodef{AGL}[AGL]{Above Ground Level}

\lstset {
	language=C,
	basicstyle=\ttfamily,
	keywordstyle=\color{blue}\ttfamily,
	stringstyle=\color{red}\ttfamily,
	commentstyle=\color{OliveGreen}\ttfamily,
	morecomment=[l][\color{magenta}]{\#}
	showstringspaces=false,
	showspaces=false,
	frame=single,
	captionpos=b
}

\newcommand{\commandline}[2][\empty] {
	\begin{quote}
		\texttt{#2}
		\ifthenelse{\equal{#1}{\empty}}{}{\begin{quote}#1\end{quote}}
	\end{quote}
}

\newcommand{\sigline}[1][\empty] {
	\vspace{1in}
	\hrule width0.5\textwidth
	\vspace{1mm}
	\noindent #1	
}

\newcommand*{\SignatureAndDate}[1]{
	\vspace{1in}
	\par\noindent\makebox[2.5in]{\hrulefill} \hspace{.5in} \makebox[2.0in]{\hrulefill}
	\par\noindent\makebox[2.5in][l]{#1}      \hspace{.5in} \makebox[2.0in][l]{Date}
}


\begin{document}
	\singlespace
	
	\title{\vspace{2in}Design}
	
	\author {
		Anisimova, Natasha
		\and
		Lee, Terrance
		\and
		Morgan, Albert
	}
	
	\markboth{CS Capstone 2016-2017}{Groundstation}
	
	\pagestyle{empty}
	\vspace*{2in}
	\begin{center}
		\huge
		Groundstation: Progress\\
		\normalsize
		\vspace{5mm}
		\textbf{
			Team \#25\\
			High-Altitude Rocketry Challenge\\
		}
		\vspace{1mm}
		Natasha Anisimova\\
		Terrance Lee\\
		Albert Morgan
	\end{center}
	
	\vspace{5mm}
	
	\begin{center}
		\textbf{Abstract}
	\end{center}
	
	%\begin{adjustwidth}{0.75in}{0.75in}
	
	%Abstract goes here
	The \textit{Groundstation} software will collect telemetry from a rocket while it is in flight and graphically display the telemetry in real-time. Groundstation is made up several different components: collection of data, storage of data, interpolation of data, and 
	display of data.
	This document will describe the progress that was made by the Groundstation software team over the course between September and December of 2016.
	%\end{adjustwidth}
		
	\pagestyle{headings}
	
	\newpage

	% Uncomment this to make the table of contents	
	\tableofcontents
	\newpage

\section{Purpose and Goals}
	In June 2017, the \ac{OSU} chapter of the
	\ac{AIAA} will launch a rocket in the Black Rock Desert.
	This rocket will ascend to one hundred thousand feet \ac{AGL}.
	Designing, building, and launching the rocket will require the
	collaboration and expertise of dozens of engineers from a variety
	of disciplines, including mechanical, electrical, computer, and
	software.
	
	Many rockets record data during the flight. This data may include
	altitude, latitude, longitude, and more.
	Latitude and longitude
	data is helpful for locating the rocket after it lands, but
	if the data is located on the rocket, then it is useless
	for this task.
	Altitude data will tell the rocketry team if we have
	met the objective of one hundred thousand feet, but if the rocket
	is damaged or cannot be located for any reason, then the record
	of this achievement may be lost.
	
	However, the chicken-and-egg problem of not being able to access
	the location data before the rocket is found may be avoided by sending
	telemetry from the rocket in real-time.
	Telemetry may be received wirelessly
	by the rocketry team while the rocket is in flight, giving them
	access to useful information such as the rocket's last known location
	and the highest altitude reached. This telemetry will aid in the
	rocket's recovery and provide a log of the rocket's flight even if
	recovery is impossible.

\section{Current Progress}
Currently, we have fully designed the system. However, very little development with has begun.
Natasha has done some proof-of-concepts with 3.js to ensure that it will work on everyone's systems,
but no work has begun on the backend or final user interface.

\section{Retrospective}

\begin{tabular}{| p{0.3\linewidth} | p{0.3\linewidth} | p{0.3\linewidth} |}
\hline
Positives & Deltas & Actions\\
\hline

& After reviewing the development log for the last ten weeks, it was clear that we barely met most of our deadlines.
Instead of doing a large amount of work at the last minute, we should do a little bit of work over the course of the time we have on an assignment.
& In order to better meet our deadlines, we will institute an internal timeline in addition to the due-dates provided by the assignment description. We will use Git issues and milestones to set deadlines and assign tasks.
These tasks will break up the whole assignment into bitesized pieces that can be completed throughout the work period.
If this is not sufficient, we will look at more feature-rich options such as waffle.io or Git projects.
\\

\hline
\end{tabular}

\section{Development Log}


\subsection{Week 1}
\subsubsection{Anisimova, Natasha}
\begin{itemize}
	\item \textbf{Plans: }
	For this next week I need to figure out the scale and load the terrain into the webpage. Colorizing it would be good as well. The Avionics and Recovery Team want to do an integration test soon as well so we need to keep that in mind. 
	\item \textbf{Progress: }
	Now that we have the height map and Al was able to get an image out of the .xtr file, I was able to get a 3D model of the 	terrain around Spaceport America. Scaling is now our biggest concern. 
	\item \textbf{Problems: }
	Figuring out the longitude and latitude scale compared to miles for the model. It seems like the 3D model is skewed a bit. The east and west were elongated slightly. The webpage needs to be re-organized as well. Having the 3D terrain be the first thing that a person sees would be ideal. 
\end{itemize}

\subsection{Week 1}
\subsubsection{Lee, Terrance}
\begin{itemize}
	\item \textbf{Plans: }
	I have a meeting with Ben Brewster who knows more about networking next week. I want to get systemd implemented into inittab. Get the layout for the line graph layout at minimum if not more so that we can get that setup. Also want to get our final link setup or at least a link setup that we can use for expo so that I have a network that I can use for the line graph setup. 
	\item \textbf{Progress: }
	I found a good resource for our Live Line Graph for the rocket and I found what was need for Systemd and inittab to work together. I also went to a meeting with avionics and recovery about a couple of things that they wanted to get covered and make sure everybody was on the right page. We were originally not needed for this meeting but I am glad I went but cause there was a few things that we should know about and somethings that we might want to talk about as well in our next meeting that we may want to put into our code. 
	\item \textbf{Problems: }
	Having a issue with line graph that I am using. I can hook it up to a network but it used to using websites but since we are using a raspberry pi as our network I have to ask others who know more about networking for help on this. This is why I have a meeting with Ben Brewster next week. 
\end{itemize}


\subsection{Week 1}
\subsubsection{Morgan, Al}
\begin{itemize}
	\item \textbf{Plans: } 
	Over the course of the next week I hope to finish the 3D interface and get it into something resembling its final form. Additionally, the avionics team is ready for integration testing, so I will ensure that our software is ready to interact with their hardware. the protocol has changed since we began to the project, so a few minor changes will need to be made before the software is ready to integrate, but since the framework of the protocol is already in place I do not believe that this will cause any major issues. 
	\item \textbf{Progress: }
	This week we began integrating the 3D model into the web interface. We took the new, higher resolution height map data and turned it into an image, which we then used to construct a 3D model (a .obj file) that we can import into the WebGL canvas using Three.js. I also got some very basic objects to move around the scene to represent the rocket. I haven't actually hooked the moving objects into the telemetry data yet, but all of the pieces are in place to do so. 
	\item \textbf{Problems: }
	We had a minor hiccup when we realized that the Earth is round and one degree of latitude does not equal one degree of longitude in terms of distance. However, some quick and easy trigonometry fixed this issue.
\end{itemize}

\subsection{Week 2}
\subsubsection{Anisimova, Natasha}
\begin{itemize}
	\item \textbf{Plans: }
	Getting the atmospheres to be at the correct height will be a problem since everything has to be scaled correctly according to the data that we are receiving. The 3.js transparency works, we just need to make sure that we have all the correct libraries for it and that it is set up correctly in the environment. 
	\item \textbf{Progress: }
	Originally the terrain that I had created was off in terms of scale to longitude and latitude compared to Google Earth. Al was able to figure out the math behind it and alter it. The terrain also had a distorted height values due to how Blender was adding the modifier for height of the height map image to the flat plane. After speaking to the rest of the mechanical engineers, they said they wanted it to be as close to the realistic area as possible. This lead us to having a quite a flat plane. I have also been working on transparency since that will be useful for showing the different atmospheres that the rocket is going through. 
	\item \textbf{Problems: } 
	
\end{itemize}

\subsection{Week 2}
\subsubsection{Lee, Terrance}
\begin{itemize}
	\item \textbf{Plans: }
	\item \textbf{Progress:  }
	\item \textbf{Problems: }
\end{itemize}


\subsection{Week 2}
\subsubsection{Morgan, Al}
\begin{itemize}
	\item \textbf{Plans: }
	\item \textbf{Progress: }
	\item \textbf{Problems: }
\end{itemize}

\subsection{Week 3}
\subsubsection{Anisimova, Natasha}
\begin{itemize}
	\item \textbf{Plans: }
	\item \textbf{Progress:  }
	\item \textbf{Problems: }
\end{itemize}

\subsection{Week 3}
\subsubsection{Lee, Terrance}
\begin{itemize}
	\item \textbf{Plans: }
	\item \textbf{Progress:  }
	\item \textbf{Problems: }
\end{itemize}


\subsection{Week 3}
\subsubsection{Morgan, Al}
\begin{itemize}
	\item \textbf{Plans: }
	\item \textbf{Progress: }
	\item \textbf{Problems: }
\end{itemize}

\subsection{Week 4}
\subsubsection{Anisimova, Natasha}
\begin{itemize}
	\item \textbf{Plans: }
	\item \textbf{Progress:  }
	\item \textbf{Problems: }
\end{itemize}

\subsection{Week 4}
\subsubsection{Lee, Terrance}
\begin{itemize}
	\item \textbf{Plans: }
	\item \textbf{Progress:  }
	\item \textbf{Problems: }
\end{itemize}


\subsection{Week 4}
\subsubsection{Morgan, Al}
\begin{itemize}
	\item \textbf{Plans: }
	\item \textbf{Progress: }
	\item \textbf{Problems: }
\end{itemize}

\subsection{Week 5}
\subsubsection{Anisimova, Natasha}
\begin{itemize}
	\item \textbf{Plans: }
	\item \textbf{Progress:  }
	\item \textbf{Problems: }
\end{itemize}

\subsection{Week 5}
\subsubsection{Lee, Terrance}
\begin{itemize}
	\item \textbf{Plans: }
	\item \textbf{Progress:  }
	\item \textbf{Problems: }
\end{itemize}


\subsection{Week 5}
\subsubsection{Morgan, Al}
\begin{itemize}
	\item \textbf{Plans: }
	\item \textbf{Progress: }
	\item \textbf{Problems: }
\end{itemize}

\subsection{Week 6}
\subsubsection{Anisimova, Natasha}
\begin{itemize}
	\item \textbf{Plans: }
	\item \textbf{Progress:  }
	\item \textbf{Problems: }
\end{itemize}

\subsection{Week 6}
\subsubsection{Lee, Terrance}
\begin{itemize}
	\item \textbf{Plans: }
	\item \textbf{Progress:  }
	\item \textbf{Problems: }
\end{itemize}


\subsection{Week 6}
\subsubsection{Morgan, Al}
\begin{itemize}
	\item \textbf{Plans: }
	\item \textbf{Progress: }
	\item \textbf{Problems: }
\end{itemize}


\end{document}
