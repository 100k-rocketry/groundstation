\documentclass[10pt,draftclsnofoot,onecolumn]{IEEEtran}
\hyphenation{op-tical net-works semi-conduc-tor}

\usepackage[margin=.75in]{geometry}
\usepackage{courier}
\usepackage{ifthen}
\usepackage{setspace}
\usepackage{listings}
\usepackage[usenames, dvipsnames]{color}
\usepackage{tabularx}
\usepackage[strict]{chngpage}
\usepackage{cite}
\usepackage{graphicx}
\usepackage{acronym}
\usepackage{color}
\usepackage{makeidx}
\usepackage{url}

\makeindex

\acrodef{OSU}[OSU]{Oregon State University}
\acrodef{AIAA}[AIAA]{American Institute of Aeronautics and Astronautics}
\acrodef{AGL}[AGL]{Above Ground Level}

\lstset {
	language=C,
	basicstyle=\ttfamily,
	keywordstyle=\color{blue}\ttfamily,
	stringstyle=\color{red}\ttfamily,
	commentstyle=\color{OliveGreen}\ttfamily,
	morecomment=[l][\color{magenta}]{\#}
	showstringspaces=false,
	showspaces=false,
	frame=single,
	captionpos=b
}

\newcommand{\commandline}[2][\empty] {
	\begin{quote}
		\texttt{#2}
		\ifthenelse{\equal{#1}{\empty}}{}{\begin{quote}#1\end{quote}}
	\end{quote}
}

\newcommand{\sigline}[1][\empty] {
	\vspace{1in}
	\hrule width0.5\textwidth
	\vspace{1mm}
	\noindent #1	
}

\newcommand*{\SignatureAndDate}[1]{
	\vspace{1in}
	\par\noindent\makebox[2.5in]{\hrulefill} \hspace{.5in} \makebox[2.0in]{\hrulefill}
	\par\noindent\makebox[2.5in][l]{#1}      \hspace{.5in} \makebox[2.0in][l]{Date}
}



\begin{document}
	\singlespace
	
	\title{\vspace{2in}Problem Statement}
	
	\author {
		Anisimova, Natasha
		\and
		Lee, Terrance
		\and
		Morgan, Albert
	}
	
	\markboth{CS Capstone 2016-2017}{Groundstation}
	
	\pagestyle{empty}
	\vspace*{2in}
	\begin{center}
		\huge
		Groundstation: Technology Review\\
		\normalsize
		\vspace{5mm}
		\textbf{
			Team \#25\\
			High-Altitude Rocketry Challenge\\
		}
		\vspace{1mm}
		Natasha Anisimova\\
		Terrance Lee\\
		Albert Morgan
	\end{center}
	
	\vspace{5mm}
	
	\begin{center}
		\textbf{Abstract}
	\end{center}
	
	%\begin{adjustwidth}{0.75in}{0.75in}
	
	Abstract goes here
	
	%\end{adjustwidth}
	
	\newpage
	
	Here's how to cite something~\cite{apache-usage-statistics}.
	Check the \LaTeX source.
	The actual sources are in groundstation.bib, add new sources as necessary.
	We can re-use this file so we don't have to rewrite it again in the future, so give your sources useful names.
	Note that if you don't cite a source from the .bib file, it doesn't show up in the references, which is why we can re-use that file.
	
	Pro-tip: put a tilde in between the cite command and the word before it to prevent \LaTeX from putting the citation on a new line.

	\newpage

	% Uncomment this to make the table of contents	
	\tableofcontents
	\newpage
	
	\pagestyle{headings}



	\section{Web server}
	\index{Web server}
	This section will describe the various options for the web server. Author: Albert Morgan.

	\subsection{Apache}
	\index{Apache}
	\index{Web server}
	% Advantages:
	% Powerful module system	
	Apache is the most popular web server option, with over 50\% of the market share~\cite{apache-usage-statistics}.
	Apache uses a powerful and flexible module system that allows for a variety of different behaviors~\cite{apache-vs-nginx-practical-considerations}
	Each connection to an Apache server spawns a new thread or process, depending on which modules are used.
	This feature has advantages and disadvantages.
	Because each connection spawns a new thread, if the thread serving one connection crashes, it won't affect the other connections.
	However, because each connection spawns a new thread, the overhead for that thread must be paid for each new connection.

	\subsection{NGINX}
	\index{NGINX}
	\index{Web server}
	% NGINX advantages:
	% fast processing
	% concurrency (don't need this, though)
	% resource efficiency
	% responsiveness
	NGINX is 2\textsuperscript{nd} most popular web server on the Internet~\cite{nginx-usage-statistics}.
	Web servers such as Apache that use a one-thread-per-connection model can have problems with large number of simultaneous connections, an issue known as the C10k problem~\cite{apache-vs-nginx-practical-considerations}~\cite{c10k-problem}.
	NGINX solves this issue by having one thread serve many connections.
	This approach reduces the overhead caused by spawning so many threads, and makes NGINX very lightweight and scalable~\cite{nginx-vs-apache-our-view}.
	
	\subsection{Lighttpd}
	\index{Lighttpd}
	\index{Web server}
	Lighttpd, pronounced ``lighty'', is a web server that is optimized for low memory footprints and fast response.
	It is optimized for serving a large number of concurrent keep-alive connections, such as serving high-volume AJAX driven web sites~\cite{lighttpd}.

	\subsection{Conclusion}
	All three web servers listed above have advantages for this project.
	Apache's one-thread-per-connection approach increases stability.
	NGINX and Lighttpd are both lightweight, which is important for running an application on a low-power system like a Raspberry Pi.
	However, Groundstation does not have to support more than 20 concurrent users, so the memory and processor savings for choosing a lightweight option will be minimal.
	Additionally, stability is a primary concern of the High-Altitude Rocketry Challenge.
	For these reasons, I recommend Apache.


	\section{Web backend}
	This section will describe the options for the web backend. Author: Albert Morgan.
	
	\subsection{Node}
	\index{Node}
	\index{V8}
	\index{JavaScript}
	\index{Google Chrome}
	\index{Chrome}
	Node allows web backends to be written in JavaScript using V8, the same engine that powers Google Chrome~\cite{node}.
	The client side of the web application will certainly use JavaScript, and writing the back-end  in JavaScript as well will reduce the overhead of learning and programming in new languages.
	Additionally, because JavaScript would be used on the front-end and back-end, it may be possible to reuse code between these two systems.
	
	Benchmarks show that Node is over 5 times as fast as PHP, even when using just-in-time compilation~\cite{comparing-node-vs-php-performance}.
	
	\subsection{PHP}
	\index{PHP}
	PHP has been around the web since 1994 and is used in over 80\% of web sites today~\cite{history-of-php}~\cite{usage-of-server-side-programming-languages-for-websites}.
	Given its maturity and wide adoption, there are a lot of packages, documentation, and example available for developers.
	
	
	\subsection{Ruby}
	\section{Persistence}
	This section will describe how the data will be stored. Author: Albert Morgan.
	\subsection{Flat files}
	\subsection{MySQL}
	\subsection{???}

	% Make the index
	\printindex


% Signatures
%\begin{minipage}{\textwidth}	
%	\SignatureAndDate{Nancy Squires}
%	\SignatureAndDate{Natasha Anisimova}
%	\SignatureAndDate{Terrance Lee}
%	\SignatureAndDate{Albert Morgan}
%\end{minipage}


\bibliography{groundstation}
\bibliographystyle{IEEEtran}

\end{document}


