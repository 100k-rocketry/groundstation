\documentclass[10pt,journal,draftclsnofoot,onecolumn]{IEEEtran}
\hyphenation{op-tical net-works semi-conduc-tor}

\usepackage[margin=.75in]{geometry}
\usepackage{courier}
\usepackage{ifthen}
\usepackage{setspace}
\usepackage{listings}
\usepackage[usenames, dvipsnames]{color}
\usepackage{tabularx}
\usepackage[strict]{chngpage}
\usepackage{cite}
\usepackage{graphicx}
\usepackage{acronym}
\usepackage{color}

\acrodef{OSU}[OSU]{Oregon State University}
\acrodef{AIAA}[AIAA]{American Institute of Aeronautics and Astronautics}
\acrodef{AGL}[AGL]{Above Ground Level}

\lstset {
	language=C,
	basicstyle=\ttfamily,
	keywordstyle=\color{blue}\ttfamily,
	stringstyle=\color{red}\ttfamily,
	commentstyle=\color{OliveGreen}\ttfamily,
	morecomment=[l][\color{magenta}]{\#}
	showstringspaces=false,
	showspaces=false,
	frame=single,
	captionpos=b
}

\newcommand{\commandline}[2][\empty] {
	\begin{quote}
		\texttt{#2}
		\ifthenelse{\equal{#1}{\empty}}{}{\begin{quote}#1\end{quote}}
	\end{quote}
}

\begin{document}
	\singlespace
	
	\title{\vspace{2in}Problem Statement}
	
	\author {
		Anisimova, Natasha
		\and
		Lee, Terrance
		\and
		Morgan, Albert
	}
	
	\markboth{CS Capstone 2016-2017}{Assignment 1}
	
	\pagestyle{empty}
	\vspace*{2in}
	\begin{center}
		\huge
		Problem Statement\\
		\normalsize
		\vspace{5mm}
		CS Capstone\\
		Spring 2016\\
		\vspace{5mm}
		Natasha Anisimova\\
		Terrance Lee\\
		Albert Morgan
	\end{center}
	
	\vspace{5mm}
	
	\begin{center}
		\textbf{Abstract}
	\end{center}
	
	\begin{adjustwidth}{2in}{2in}
		In June 2017, the OSU chapter of the
		AIAA will launch a rocket in the Mojave Desert.
		They will be the first university team to launch a rocket into the stratosphere, one hundred thousand feet above 
		sea level.
		Our team will develop the tracking software that receives
		telemetry from the rocket during its flight
		and displays the data in real-time for the entire OSU rocketry team.
		This software must be flexible, robust, and accurate in order
		to ensure that the variety of engineers on the project all receive
		the data they need on launch data.
		This document will give an overview of the problem,
		our proposed solution,
		and describe how we will measure its success.
	\end{adjustwidth}
	
	\newpage
	\pagestyle{headings}
	\section{Problem Definition}
	In June 2017, the \ac{OSU} chapter of the
	\ac{AIAA} will launch a rocket in the Mojave Desert.
	This rocket will ascend to one hundred thousand feet \ac{AGL}.
	Designing, building, and launching the rocket will require the
	collaboration and expertise of dozens of engineers from a variety
	of disciplines, including mechanical, electrical, computer, and
	software.

	Many rockets record data during the flight. This data may include
	altitude, latitude, longitude, and more.
	Latitude and longitude
	data is helpful for locating the rocket after it lands, but
	if the data is located on the rocket, then it is useless
	for this task.
	Altitude data will tell the rocketry team if we have
	met the objective of one hundred thousand feet, but if the rocket
	is damaged or cannot be located for any reason, then the record
	of this achievement may be lost.

	However, the chicken-and-egg problem of not being able to access
	the location data before the rocket is found may be avoided by sending
	telemtry from the rocket in real-time.
	Telemetry may be received wirelessly
	by the rocketry team while the rocket is in flight, giving them
	access to useful information such as the rocket's last known location
	and the highest altitude reached. This telemetry will aid in the
	rocket's recovery and provide a log of the rocket's flight even if
	recovery is impossible.
	
	% Use the \subsection command to create a subsection:
	% \subsection{Section title}
	\section{Proposed Solution}

	\subsection{The Ground Station}

	With the aid of the computer engineers in the avionics team,
	the software team will design, build, and deploy a ground station
	capable of receiving and displaying essential telemetry in real-time.
	At a minimum, this telemetry includes highest recorded altitude and
	last known latitude and longitude. In addition to displaying
	the data in real-time, this data should be logged so it can
	be accessed later.

	\subsection{Desired Features}

	After the primary objectives outlined in the previous section
	are met, there are a number of optional features
	that have been requested by the \ac{OSU} rocketry team.
	Because the reliability of this software is essential on launch day,
	additional features will only be implemented after
	the stability, robustness, and accuracy of previous features
	is ensured.

	A customized web server will be developed for the ground station,
	and the rocketry team will be able to connect to the ground station
	though a wireless access point. Once connected, data will be able
	to be viewed in real-time in a graphical format through their
	web browser.

	The software team will work closely with the engineers in the
	rest of the rocketry team to ensure the software
	meets their needs on launch day. These features will be considered
	and prioritized at a later date, but should never compromise the stability,
	robustness, and accuracy of the system or its ability
	to display the essential telemetry.
	
	\subsection{Launch Simulation}
	
	The mechanical engineers of the group have requested a launch simulator that
	would allow them to control sensors on-board the rocket in order to test
	systems that rely on certain sensor data. This will consist of a separate
	software package that will only be designed and implemented after the
	essential features of the ground station are complete. The launch simulator
	will be prioritized along with the other desired features at a later date.

	\subsection{Expo}
	At the engineering expo, the software team will demonstrate their software
	in a simulated environment. This may include replaying real data from test launches
	or using hand-crafted data.
	If the desired web server function is complete, then people will be given the ability
	to connect to our server with their personal computers and view the data themselves.
	Ideally, this task would be done by working closely
	with the avionics team to highlight the combined achievements of our
	hardware and software.
	
	\section{Performance Metrics}
	
	The successfulness of the project will measured by whether the data
	is successfully recorded and made accessible to the rocketry team
	on launch day. To ensure the software is ready for launch day,
	we will consider three primary performance metrics:
	
	\begin{itemize}
		\item Stability: the software should not crash or otherwise cease operation
		\item Robustness: the software should handle bad inputs gracefully
		\item Accuracy: the software should correctly report all telemetry
	\end{itemize}
	
	In order to meet these requirements, a comprehensive test plan will
	be written and executed. This plan will involve sending large amounts
	of random, recorded, and live data to the software and recording the results.
	The software will only be considered successful if it passes all of these
	tests with no errors.

	\vspace{1in}
	\noindent Nancy Squires

	\vspace{1in}
	\noindent Natasha Anisimova

	\vspace{1in}
	\noindent Terrance Lee

	\vspace{1in}
	\noindent Albert Morgan\\
	






\end{document}


