\documentclass[10pt,journal,draftclsnofoot,onecolumn]{IEEEtran}
\hyphenation{op-tical net-works semi-conduc-tor}

\usepackage[margin=.75in]{geometry}
\usepackage{courier}
\usepackage{ifthen}
\usepackage{setspace}
\usepackage{listings}
\usepackage[usenames, dvipsnames]{color}
\usepackage{tabularx}
\usepackage[strict]{chngpage}
\usepackage{cite}
\usepackage{graphicx}
\usepackage{acronym}


\acrodef{OSU}[OSU]{Oregon State University}
\acrodef{AIAA}[AIAA]{American Institute of Aeronautics and Astronautics}
\acrodef{AGL}[AGL]{Above Ground Level}

\lstset {
	language=C,
	basicstyle=\ttfamily,
	keywordstyle=\color{blue}\ttfamily,
	stringstyle=\color{red}\ttfamily,
	commentstyle=\color{OliveGreen}\ttfamily,
	morecomment=[l][\color{magenta}]{\#}
	showstringspaces=false,
	showspaces=false,
	frame=single,
	captionpos=b
}

\newcommand{\commandline}[2][\empty] {
	\begin{quote}
	\texttt{#2}
	\ifthenelse{\equal{#1}{\empty}}{}{\begin{quote}#1\end{quote}}
	\end{quote}
}

\begin{document}
\singlespace

\title{\vspace{2in}Problem Statement}

\author {
	Anisimova, Natasha
	\and
	Lee, Terrance
	\and
	Morgan, Albert
}

\markboth{CS Capstone 2016-2017}{Assignment 1}

\pagestyle{empty}
\vspace*{2in}
\begin{center}
\huge
Problem Statement\\
\normalsize
\vspace{5mm}
CS Capstone\\
Spring 2016\\
\vspace{5mm}
Natasha Anisimova\\
Terrance Lee\\
Albert Morgan
\end{center}

\vspace{5mm}

\begin{center}
\textbf{Abstract}
\end{center}

\begin{adjustwidth}{2in}{2in}
In June 2017, the \ac{OSU} chapter of the
American Institute of Aeronautics and Astronautics will launch a rocket in the Nevada desert.
We will be the first university team to to launch a rocket past the boundary into space,
known as the the K\'{a}rm\'{a}n line, which lies 100km above the surface of the Earth.
Our team will develop the tracking software that receives
telemetry from the rocket during its flight
and displays the data in real-time for the engineering team.
The software must be flexible enough to work with a variety of sensors,
not all of which have been decided on yet.
Additionally, the software must be robust,
because there will be no second chances on launch day.
This document will give an overview of the problem,
our proposed software solution,
and describe how we will measure its success.
\end{adjustwidth}

\newpage
\pagestyle{headings}
\section{Problem Definition}
In June 2017, the \ac{OSU} chapter of the
\ac{AIAA} will launch a rocket in the Nevada desert.
This rocket will ascend to one hundred thousand feet \ac{AGL}.
Designing, building, and launching the rocket will require the
collaboration and expertise of dozens of engineers from a variety
of disciplines, including mechanical, electrical, computer, and
software.

While the rocket is in flight, the \ac{OSU} rocketry team
must be able to receive, view, and store telemetry in real-time.
Mission critical information that must be received by the ground
station are latitude, longitude, and altitude. The latitude
and longitude will aid in tracking and recovery of the rocket
after flight. Real-time altitude readings will allow the rocket
team to determine if the objective of one hundred thousand feet
has been met even if recovery of the rocket is impossible.


% Use the \subsection command to create a subsection:
% \subsection{Section title}
\section{Proposed Soltion}


\subsection{The Ground Station}

The final software package must be stable and resistance to errors.
Losing peak altitude data could cause the record of the team's
achievement to be lost, and any error in the latitude
and longitude records could severely hamper or derail
the recovery of the rocket.

\subsection{Stretch Goals}
After the primary objectives outlined in the previous section
are met, there are a number of additional optional features
that have been requested by the rocketry team.
Many of our clients are still new to the team, and the specific
needs and wants may change as the project progresses.
Stretch goals may include, in no particular order:

\begin{itemize}
\item The engineers would like to be able to record and view data
from a number of different sensors, not just altitude,
latitude, and longitude.

\item The software package could present the data graphically.
This could include line graphs, overlaying position data
on maps, or more.

\item The data could be made availble 

\item A configurable interface could allow team members
to select the data that they want to view in real-time.

\item Automatic uploading of data to a remote host.

\end{itemize}

The Computer Science team is 
working closely with the ECE team, we are considering having two servers 
present at the ground station. One would exist as a web server, this would
serve and display data about the rocket on a local Wi-Fi network. We will
write customized software that retrieves data from an external source and
formats it before making it available to the web server. The second server,
 or application server, is where that software will exist. This server will
 read data from the serial port and log it. Since the Rocket Team hopes to 
 use this software in future launches we plan on making it easy to install.

\subsection{The Ground Station}

The final software package must be stable and resistance to errors.
Losing peak altitude data could cause the record of the team's
achievement to be lost, and any error in the latitude
and longitude records could severely hamper or derail
the recovery of the rocket.


Put stuff we \textit{have} to complete here.

Add another section for stretch goals (but think of a better name).

Simulations?


\section{Performance Metrics}

The successfulness of the project will measured by whether the data
is successfully recorded and made accessible to the rocketry team
on launch day. To ensure the software is ready for launch day,
we will consider three primary performance metrics:

\begin{itemize}
\item Stability: the software should not crash or otherwise cease operation
\item Robustness: the software should handle bad inputs gracefully
\item Accuracy: the software should correctly report all telemetry
\end{itemize}

In order to meet these requirements, a comprehensive test plan will
be written an executed. This plan will involve sending large amounts
of random, recorded, and live data to the software, including:

\begin{itemize}

\item Completely random data
\item Valid data with random bits flipped
\item Data with valid structure but invalid 
\item Randomly generated valid data
\item Test launch data
\item Real data collected from the transmitter (not during launch)
\end{itemize}

The software will only be considered successful if it passes all of these
tests with no errors.

\end{document}


