\documentclass[letterpaper,10pt,draftclsnofoot,onecolumn]{IEEEtran}
\usepackage{graphicx}                                        
\usepackage{amssymb}                                         
\usepackage{amsmath}                                         
\usepackage{amsthm}                                          

\usepackage{alltt}                                           
\usepackage{float}
\usepackage{color}
\usepackage{url}

\usepackage{balance}
\usepackage[TABBOTCAP, tight]{subfigure}
\usepackage{enumitem}
\usepackage{pstricks, pst-node}

\usepackage{geometry}
\geometry{textheight=8.5in, textwidth=6in}

\newcommand{\cred}[1]{{\color{red}#1}}
\newcommand{\cblue}[1]{{\color{blue}#1}}

\newcommand{\toc}{\tableofcontents}

\def\name{Natasha Anisimova, Terrance Lee, Albert Morgan}
\parindent = 0.0 in
\parskip = 0.1 in
\title{Problem Statement}
\date{October 3, 2016}
\author{Natasha Anisimova, Terrance Lee, Albert Morgan
	
	October 3, 2016}
\begin{document}

\maketitle
	\tableofcontents


\begin{abstract}
In June 2017, the Oregon State University chapter of the
American Institute of Aeronautics and Astronautics will launch a rocket in the Nevada desert.
We will be the first university team to to launch a rocket past the boundary into space,
known as the the K\'{a}rm\'{a}n line.
Our team will develop the tracking software that receives
telemetry from the rocket as it makes its ascent.
\end{abstract}
	
	\section{Introduction}
	
	
\end{document}
