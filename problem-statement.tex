\documentclass[10pt,journal,draftclsnofoot,onecolumn]{IEEEtran}
\hyphenation{op-tical net-works semi-conduc-tor}

\usepackage[margin=.75in]{geometry}
\usepackage{courier}
\usepackage{ifthen}
\usepackage{setspace}
\usepackage{listings}
\usepackage[usenames, dvipsnames]{color}
\usepackage{tabularx}
\usepackage[strict]{chngpage}
\usepackage{cite}
\usepackage{graphicx}

\lstset {
	language=C,
	basicstyle=\ttfamily,
	keywordstyle=\color{blue}\ttfamily,
	stringstyle=\color{red}\ttfamily,
	commentstyle=\color{OliveGreen}\ttfamily,
	morecomment=[l][\color{magenta}]{\#}
	showstringspaces=false,
	showspaces=false,
	frame=single,
	captionpos=b
}

\newcommand{\commandline}[2][\empty] {
	\begin{quote}
	\texttt{#2}
	\ifthenelse{\equal{#1}{\empty}}{}{\begin{quote}#1\end{quote}}
	\end{quote}
}

\begin{document}
\singlespace

\title{\vspace{2in}Problem Statement}

\author {
	Anisimova, Natasha
	\and
	Lee, Terrance
	\and
	Morgan, Albert
}

\markboth{CS Capstone 2016-2017}{Assignment 1}

\pagestyle{empty}
\vspace*{2in}
\begin{center}
\huge
Problem Statement\\
\normalsize
\vspace{5mm}
CS Capstone\\
Spring 2016\\
\vspace{5mm}
Natasha Anisimova\\
Terrance Lee\\
Albert Morgan
\end{center}

\vspace{5mm}

\begin{center}
\textbf{Abstract}
\end{center}

\begin{adjustwidth}{2in}{2in}
In June 2017, the Oregon State University chapter of the
American Institute of Aeronautics and Astronautics will launch a rocket in the Nevada desert.
We will be the first university team to to launch a rocket past the boundary into space,
known as the the K\'{a}rm\'{a}n line, which lies 100km above the surface of the Earth.
Our team will develop the tracking software that receives
telemetry from the rocket during its flight
and displays the data in real-time for the engineering team.
The software must be flexible enough to work with a variety of sensors,
not all of which have been decided on yet.
Additionally, the software must be robust,
because there will be no second chances on launch day.
This document will give an overview of the problem,
our proposed software solution,
and describe how we will measure its success.
\end{adjustwidth}

\newpage
\pagestyle{headings}
\section{Problem Definition}
% Use the \subsection command to create a subsection:
% \subsection{Section title}
Write the problem definition here.
\section{Proposed Soltion}
Write the proposed solution here.
\section{Performance Metrics}
Write the performance metrics here.
\end{document}


